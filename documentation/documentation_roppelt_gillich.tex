\documentclass{sigchi-ext}
% Please be sure that you have the dependencies (i.e., additional
% LaTeX packages) to compile this example.
\usepackage[T1]{fontenc}
\usepackage[utf8]{inputenc}
\usepackage{textcomp}
\usepackage[scaled=.92]{helvet} % for proper fonts
\usepackage{graphicx} % for EPS use the graphics package instead
\usepackage{balance}  % for useful for balancing the last columns
\usepackage{booktabs} % for pretty table rules
\usepackage{ccicons}  % for Creative Commons citation icons
\usepackage{ragged2e} % for tighter hyphenation
\usepackage{subcaption}
\usepackage{caption}

% \usepackage{marginnote} \usepackage[shortlabels]{enumitem}
% \usepackage{paralist}

%% EXAMPLE BEGIN -- HOW TO OVERRIDE THE DEFAULT COPYRIGHT STRIP --
\copyrightinfo{}
%Permission to make digital or hard copies of all or}
% part of this work for personal or classroom use is granted without
% fee provided that copies are not made or distributed for profit or
% commercial advantage and that copies bear this notice and the full
% citation on the first page. Copyrights for components of this work
% owned by others than ACM must be honored. Abstracting with credit is
% permitted. To copy otherwise, or republish, to post on servers or to
% redistribute to lists, requires prior specific permission and/or a
% fee. Request permissions from permissions@acm.org.\\
% {\emph{CHI'14}}, April 26--May 1, 2014, Toronto, Canada. \\
% Copyright \copyright~2014 ACM ISBN/14/04...\$15.00. \\
% DOI string from ACM form confirmation}
%% EXAMPLE END

\title{Dokumentation zum App Design\\\small{Designworkshop 2, Sommersemester 2016}}

\numberofauthors{2}
% Notice how author names are alternately typesetted to appear ordered
% in 2-column format; i.e., the first 4 autors on the first column and
% the other 4 auhors on the second column. Actually, it's up to you to
% strictly adhere to this author notation.
\author{%
  \alignauthor{%
    \textbf{Bianka Roppelt}\\
    \affaddr{LMU München} \\
    \affaddr{München, Deutschland} \\
    \affaddr{roppelt@cip.ifi.lmu.de} }\alignauthor{%
    \textbf{Jan Gillich}\\
    \affaddr{LMU München}\\
    \affaddr{München, Deutschland}\\
    \email{gillich@cip.ifi.lmu.de} } \vfil 
}
% Paper metadata (use plain text, for PDF inclusion and later
% re-using, if desired)
\def\plaintitle{Dokumentation zum App Design im Designworkshop 2, Sommersemester 2016} \def\plainauthor{Bianka Roppelt, Jan Gillich}
\def\plainkeywords{Festival für unangepasste Kunst; Designworkshop; App Design; Medieninformatik}
\def\plaingeneralterms{Dokumentation}

%% Set up our PDF with metadata
\hypersetup{%
  pdftitle={\plaintitle}, pdfauthor={\plainauthor},
  pdfkeywords={\plainkeywords}, }

% \reversemarginpar%
\renewcommand{\figurename}{Abbildung}
\renewcommand{\abstractname}{Kurzzusammenfassung}
\begin{document}

\maketitle

% Uncomment to disable hyphenation (not recommended)
% https://twitter.com/anjirokhan/status/546046683331973120
\RaggedRight{} 

% Do not change the page size or page settings.
\begin{abstract}
Dieses Dokument beschreibt verschiedene Aspekte unserer für den Designworkshop gestalteten und entwickelten App. Außerdem erklären wir den Weg von den Ideen zur fertig Anwendung.
\end{abstract}

\keywords{\plainkeywords}

%\category{H.5.m}{Information interfaces and presentation (e.g.,
 % HCI)}{Miscellaneous}\category{See}{\url{http://acm.org/about/class/1998/}}{for
  %full list of ACM classifiers. This section is required.}

\section{Einleitung}
Im Rahmen des Designworkshops 2 an der LMU München im Sommersemester 2016 haben wir eine App gestaltet und entwickelt, die das Festival der unangepassten Kunst begleiten sollte. Ziel der App sollte es sein, Informationen zum Festival zu liefern und Eindrücke festzuhalten. Der Gestaltungsprozess, einzelne Facetten des Designs und die fertige App sollen im Folgenden dokumentiert werden.
\section{Designprozess}
blabla
\subsection{Anforderungen}
\subsection{Skizzen und Paper Prototype}
\subsection{Klick Prototyp}
\subsection{Entwicklung}
\section{Ansichten und Funktionen}
Im Folgenden werden die Funktionen der App erläutert. Alle Ansichten sind in Abbildung \ref{fig:overview} dargestellt.

\subsection{Auswahl der Festival-Wochenenden}
Beim Starten der Anwendung erreicht man die Ansicht zur Auswahl eines Festival-Wochenendes. Für jedes Wochenende gibt es eine Seite, zwischen welchen durch das Wischen nach links und rechts gewechselt werden kann. Diese Seiten enthalten ein Bild in der Farbe des jeweiligen Wochenendes, sowie die wichtigsten Informationen. Durch das Auswählen einer dieser Seiten wird man zur Übersicht des Wochenendes weitergeleitet.

\subsection{Hinweis beim ersten Start der Anwendung}
Beim ersten Start der Anwendung wird ein Hinweis  mit kurzen Informationen zum Nutzen der App angezeigt. Durch das Klicken auf einen Button wird gespeichert, dass der Hinweis in Zukunft nicht mehr angezeigt wird.

\subsection{Übersicht eines Festival-Wochenendes}
Die Übersicht eines Festival-Wochenendes besteht aus der Werke Übersicht, sowie aus der Künstler Übersicht. Des Weiteren gibt es ein Menü in der \textit{Toolbar}, durch welches man auf weitere Ansichten mit zusätzlichen Informationen gelangen kann (vgl. Abschnitt \nameref{sec:navigation}).

\subsection{Werke Übersicht}
Alle Werke des ausgewählten Wochenendes sind in der Werke-Übersicht zufällig durchmischt dargestellt (siehe Abbildung \ref{fig:werke}). Bei Klick auf ein Werk öffnet sich die Detailansicht zu diesem Werk.

\begin{figure}[h]
    \centering
        \includegraphics[width=.25\textwidth]{figures/werke.png}
    \caption{Werke Übersicht}
    \label{fig:werke}
\end{figure}

\subsection{Künstler Übersicht}
Jeder Künstler des Festival-Wochenendes ist in der Übersicht mit Profilbild und handgeschriebenem Namen dargestellt. Durch das Auswählen eines Künstlers kann man auf der Künstler Detailansicht weitere Werke betrachten und mehr Informationen zu diesem Künstler erhalten.

\subsection{Werke Detailansicht}
Die Werke Detailansicht besteht aus einem \textit{View Pager}. Das bedeutet, dass durch das \textit{Swipen} sofort andere Werke erscheinen. In der Detailansicht wird lediglich das Werk auf schwarzem Hintergrund angezeigt, welches durch \textit{Zoomen} vergrößert werden kann. Durch Klicken auf das Werk erscheinen die Kontrollelemente der Navigation, sowie der zugehörende Künstler. Durch einen Klick auf den Künstlernamen erreicht man die Künstler Detailansicht.

\subsection{Künstler Detailansicht}
Die Künstler Detailansicht beinhaltet einen Text über den Künstler, sowie Bilder von dessen Werken. Ist der Text zu lange, wird nur ein Teil angezeigt, wobei der ganze Text durch das Ausklappen der Karte angezeigt werden kann. Bei der Auswahl eines Werks des Künstlers wird die Werk Detailansicht geöffnet, wobei durch das Swipen nur die Werke dieses Künstlers angezeigt werden. In der \textit{Toolbar} wird - wie in der Künstler Übersicht - das Profilbild des Künstlers, sowie dessen Name angezeigt. 

\subsection{Anfahrt}
In der Anfahrt Ansicht wird eine Google Maps Karte\footnote{\url{https://maps.google.com/}} mit drei Marken an den Orten in München angezeigt. Jeder Marker ist in der jeweiligen Farbe des Wochenendes (vgl. Abschnitt \nameref{sec:visual}). Beim Klick auf einen der Marker erscheint eine Karte mit Informationen über dieses Event (Name, Datum und Ort des ausgewählten Festival Wochenendes). 

\subsection{Event Informationen}
Zu den Event Informationen gehören das Teaser-Video, eine Beschreibung des Festivals, Termine und Orte, sowie die Gründer des Festivals. Diese Informationen sind über die Ansicht der Event Informationen abrufbar. Da diese Daten viel Platz in Anspruch nehmen, wurden die Beschreibung und die Termine des Festivals in aufklappbare Karten integriert. Die Gründer bestehen aus denselben Elementen wie in der Künstler Übersicht. Durch einen Klick auf eines der Elemente kann man in der Künstler Detailansicht mehr über diesen Gründer erfahren.

\subsection{Impressum}
Im Impressum sind Informationen über die Herkunft der Informationen, den Aufbau der Anwendung, sowie rechtliche Absicherungen festgehalten.


\subsection{Navigation}
\label{sec:navigation}
Die Navigation durch die Anwendung wird in Abbildung \ref{fig:overview} dargestellt. Das Menü in der \textit{Toolbar} ist von der Ansicht zur Auswahl eines Festival-Wochenendes, sowie von der Übersicht eines Festival-Wochenendes aus erreichbar. Dadurch hat der Nutzer die Möglichkeit auf die Anfahrtsansicht, die Ansicht der Event Informationen, sowie das Impressum zu gelangen. Hat der Nutzer ein Wochenende ausgewählt, kann er stets durch den Klick auf das App-Icon in der Toolbar oder auf das Menü-Item "`Events"' zurück zur Übersicht der Festival-Wochenenden geleitet werden. 

\begin{figure*}[p]
    \centering
        \includegraphics[width=.85\textwidth]{figures/overview.png}
    \caption{Navigationsfluss aller Ansichten}
    \label{fig:overview}
\end{figure*}
\section{Visuelle Gestaltung}
\label{sec:visual}
Bei der visuellen Gestaltung der Anwendung haben wir uns an folgenden Grundsätzen orientiert:
\paragraph{Zurückhaltende Visualität:}Für uns war es wichtig, die angezeigten Kunstwerke gut zu präsentieren. Deswegen sollte die App visuell nicht aufdringlich sein und selbst in den Hintergrund treten. Dazu haben wir uns für entsättigte und gedeckte Farben entschieden: die Primärfarbe ist ein dunkles Grau, die Hintergrundfarbe ein leicht abgeschwächtes Weiß. Alle Animationen sind dezent und lenken nicht vom Inhalt der Anwendung ab.
\paragraph{Einhalten von Design Guidelines} Um das Vorwissen der Nutzer über die Bedienung und Bedeutung verschiedener Elemente zu nutzen haben wir versucht, soweit wie möglich die Android Material Design Guidelines\footnote{\url{https://material.google.com/}} zu befolgen. Dazu gehört zum Beispiel das Einsetzen sogenannter \textit{Cards}, eine konsistente Benutzung von \textit{Up-Arrows} und das Anwenden bewährter Design-Muster (z.B. \textit{View Pager} und \textit{Tab Layouts}).
\paragraph{Wiederverwenden von Elementen des Festivals} Begleitend zum Festival für unangepasste Kunst wurde ein Katalog erstellt um die verschiedenen Werke und Künstler zu dokumentieren. Hieraus entstanden einige Elemente, die wir in der App wiederverwendet haben, um eine Verbindung zu schaffen. Dazu gehören zum Beispiel handgeschriebene Namen der Künstler und verschiedene Grafiken, die als Titelbild für die einzelnen Wochenenden verwendet wurden. Außerdem wurden von den Veranstaltern des Festivals drei Farben zur Repräsentation der einzelnen Wochenenden festgelegt: Rot, Schwarz und Gelb. Diese flossen bei uns in die Akzentfarbe und Gestaltung der Titelbilder und der Markierungen auf der Karte mit ein. Ein interessantes Detail kann man auch im Logo-Design erkennen. Das Logo selbst besteht aus drei Komponenten in den drei Farben. Je nach ausgewähltem Wochenende ist immer der entsprechende Teil oben und steht leicht hab (vgl. Abb.~\ref{fig:icons}).
Zusätzlich haben wir versucht verschiedene Elemente ein wenig "unangepasst" zu gestalten, ohne gegen die oben genannten Grundsätze zu verstoßen. Als Beispiel kann man die zufällig schräg abgeschnittenen Profilbilder der Künstler nennen.

\begin{figure}
    \centering
    \begin{subfigure}[t]{0.3\columnwidth}
        \includegraphics[width=\textwidth]{figures/i_demokratie}
        \caption{Demokratie}
        \label{fig:demo}
    \end{subfigure}
    \begin{subfigure}[t]{0.3\columnwidth}
        \includegraphics[width=\textwidth]{figures/i_macht}
        \caption{Macht}
        \label{fig:macht}
    \end{subfigure}
    \begin{subfigure}[t]{0.3\columnwidth}
        \includegraphics[width=\textwidth]{figures/i_partizipation}
        \caption{Partizipation}
        \label{fig:parti}
    \end{subfigure}
    \caption{Icon je nach Festival-Wochenende}
    \label{fig:icons}
\end{figure}
%\input{section_technical}
%\input{section_conclusion}

\balance{} 

% \bibliographystyle{ACM-Reference-Format-Journals}
\bibliographystyle{SIGCHI-Reference-Format}
% \bibliographystyle{acm}
\bibliography{sample}

\end{document}

%%% Local Variables:
%%% mode: latex
%%% TeX-master: t
%%% End:
