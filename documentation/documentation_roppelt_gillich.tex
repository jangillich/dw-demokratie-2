\documentclass{sigchi-ext}
% Please be sure that you have the dependencies (i.e., additional
% LaTeX packages) to compile this example.
\usepackage[T1]{fontenc}
\usepackage{textcomp}
\usepackage[scaled=.92]{helvet} % for proper fonts
\usepackage{graphicx} % for EPS use the graphics package instead
\usepackage{balance}  % for useful for balancing the last columns
\usepackage{booktabs} % for pretty table rules
\usepackage{ccicons}  % for Creative Commons citation icons
\usepackage{ragged2e} % for tighter hyphenation

% \usepackage{marginnote} \usepackage[shortlabels]{enumitem}
% \usepackage{paralist}

%% EXAMPLE BEGIN -- HOW TO OVERRIDE THE DEFAULT COPYRIGHT STRIP --
% \copyrightinfo{Permission to make digital or hard copies of all or
% part of this work for personal or classroom use is granted without
% fee provided that copies are not made or distributed for profit or
% commercial advantage and that copies bear this notice and the full
% citation on the first page. Copyrights for components of this work
% owned by others than ACM must be honored. Abstracting with credit is
% permitted. To copy otherwise, or republish, to post on servers or to
% redistribute to lists, requires prior specific permission and/or a
% fee. Request permissions from permissions@acm.org.\\
% {\emph{CHI'14}}, April 26--May 1, 2014, Toronto, Canada. \\
% Copyright \copyright~2014 ACM ISBN/14/04...\$15.00. \\
% DOI string from ACM form confirmation}
%% EXAMPLE END

\title{Dokumentation zum App Design\\\small{Designworkshop 2, Sommersemester 2016}}

\numberofauthors{2}
% Notice how author names are alternately typesetted to appear ordered
% in 2-column format; i.e., the first 4 autors on the first column and
% the other 4 auhors on the second column. Actually, it's up to you to
% strictly adhere to this author notation.
\author{%
  \alignauthor{%
    \textbf{Bianka Roppelt}\\
    \affaddr{University of Munich} \\
    \affaddr{Munich, Germany} \\
    \affaddr{roppelt@cip.ifi.lmu.de} }\alignauthor{%
    \textbf{Jan Gillich}\\
    \affaddr{University of Munich}\\
    \affaddr{Munich, Germany}\\
    \email{gillich@cip.ifi.lmu.de} } \vfil 
}
% Paper metadata (use plain text, for PDF inclusion and later
% re-using, if desired)
\def\plaintitle{Dokumentation zum App Design im Designworkshop 2, Sommersemester 2016} \def\plainauthor{Bianka Roppelt, Jan Gillich}
\def\plainkeywords{Unangepasste Kunst; App Design}
\def\plaingeneralterms{Documentation}

%% Set up our PDF with metadata
\hypersetup{%
  pdftitle={\plaintitle}, pdfauthor={\plainauthor},
  pdfkeywords={\plainkeywords}, }

% \reversemarginpar%

\begin{document}

\maketitle

% Uncomment to disable hyphenation (not recommended)
% https://twitter.com/anjirokhan/status/546046683331973120
\RaggedRight{} 

% Do not change the page size or page settings.
\begin{abstract}
  UPDATED---\today. This sample paper describes the formatting
  requirements for SIGCHI Extended Abstract Format, and this sample
  file offers recommendations on writing for the worldwide SIGCHI
  readership. Please review this document even if you have submitted
  to SIGCHI conferences before, as some format details have changed
  relative to previous years. Abstracts should be about 150
  words. Required.
\end{abstract}

\keywords{\plainkeywords}

%\category{H.5.m}{Information interfaces and presentation (e.g.,
 % HCI)}{Miscellaneous}\category{See}{\url{http://acm.org/about/class/1998/}}{for
  %full list of ACM classifiers. This section is required.}

\section{Introduction}
This format is to be used for submissions that are published in the
conference publications. We wish to give this volume a consistent,
high-quality appearance. We therefore ask that authors follow some
simple guidelines. In essence, you should format your paper exactly
like this document. The easiest way to do this is to replace the
content with your own material.
\input{section2}

\balance{} 

% \bibliographystyle{ACM-Reference-Format-Journals}
\bibliographystyle{SIGCHI-Reference-Format}
% \bibliographystyle{acm}
\bibliography{sample}

\end{document}

%%% Local Variables:
%%% mode: latex
%%% TeX-master: t
%%% End:
