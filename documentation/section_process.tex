\section{Designprozess}
Im Allgemeinen wurde ein iterativer Designprozess verfolgt. Einzelne Designentscheidungen wurden immer im Team und teilweise mit Außenstehenden überprüft. Als erstes mussten wir uns über die Anforderungen an die App klar werden. Anschließend wurden Ideen für die Gestaltung auf Papier skizziert und validiert. Diese dienten dann als Grundlage für Wireframes, die in einem Klick-Prototyp verknüpft wurden. Zuletzt wurde eine lauffähige App implementiert. 
\subsection{Anforderungen}
Ziel der App sollte es sein, Interessenten des Festivals über bevorstehende Veranstaltungen zu informieren und Besuchern des Festivals die Möglichkeit zu geben, gesehene Eindrücke wiederaufleben zu lassen. Wir haben deshalb folgende Inhalte als wichtig erachtet:
\begin{itemize}
\itemsep0.5pt
\item Profile der ausstellenden Künstler
\item Eine Auswahl an Werken
\item Informationen über Zeit und Ort der verschiedenen Veranstaltungen
\end{itemize}
\subsection{Skizzen und Paper Prototype}
\subsection{Klick-Prototyp}
\subsection{Entwicklung}